% Options for packages loaded elsewhere
\PassOptionsToPackage{unicode}{hyperref}
\PassOptionsToPackage{hyphens}{url}
%
\documentclass[
]{article}
\usepackage{lmodern}
\usepackage{amssymb,amsmath}
\usepackage{ifxetex,ifluatex}
\ifnum 0\ifxetex 1\fi\ifluatex 1\fi=0 % if pdftex
  \usepackage[T1]{fontenc}
  \usepackage[utf8]{inputenc}
  \usepackage{textcomp} % provide euro and other symbols
\else % if luatex or xetex
  \usepackage{unicode-math}
  \defaultfontfeatures{Scale=MatchLowercase}
  \defaultfontfeatures[\rmfamily]{Ligatures=TeX,Scale=1}
\fi
% Use upquote if available, for straight quotes in verbatim environments
\IfFileExists{upquote.sty}{\usepackage{upquote}}{}
\IfFileExists{microtype.sty}{% use microtype if available
  \usepackage[]{microtype}
  \UseMicrotypeSet[protrusion]{basicmath} % disable protrusion for tt fonts
}{}
\makeatletter
\@ifundefined{KOMAClassName}{% if non-KOMA class
  \IfFileExists{parskip.sty}{%
    \usepackage{parskip}
  }{% else
    \setlength{\parindent}{0pt}
    \setlength{\parskip}{6pt plus 2pt minus 1pt}}
}{% if KOMA class
  \KOMAoptions{parskip=half}}
\makeatother
\usepackage{xcolor}
\IfFileExists{xurl.sty}{\usepackage{xurl}}{} % add URL line breaks if available
\IfFileExists{bookmark.sty}{\usepackage{bookmark}}{\usepackage{hyperref}}
\hypersetup{
  pdftitle={DATA 605 - Discussion \#13},
  pdfauthor={Zach Alexander},
  hidelinks,
  pdfcreator={LaTeX via pandoc}}
\urlstyle{same} % disable monospaced font for URLs
\usepackage[margin=1in]{geometry}
\usepackage{color}
\usepackage{fancyvrb}
\newcommand{\VerbBar}{|}
\newcommand{\VERB}{\Verb[commandchars=\\\{\}]}
\DefineVerbatimEnvironment{Highlighting}{Verbatim}{commandchars=\\\{\}}
% Add ',fontsize=\small' for more characters per line
\usepackage{framed}
\definecolor{shadecolor}{RGB}{248,248,248}
\newenvironment{Shaded}{\begin{snugshade}}{\end{snugshade}}
\newcommand{\AlertTok}[1]{\textcolor[rgb]{0.94,0.16,0.16}{#1}}
\newcommand{\AnnotationTok}[1]{\textcolor[rgb]{0.56,0.35,0.01}{\textbf{\textit{#1}}}}
\newcommand{\AttributeTok}[1]{\textcolor[rgb]{0.77,0.63,0.00}{#1}}
\newcommand{\BaseNTok}[1]{\textcolor[rgb]{0.00,0.00,0.81}{#1}}
\newcommand{\BuiltInTok}[1]{#1}
\newcommand{\CharTok}[1]{\textcolor[rgb]{0.31,0.60,0.02}{#1}}
\newcommand{\CommentTok}[1]{\textcolor[rgb]{0.56,0.35,0.01}{\textit{#1}}}
\newcommand{\CommentVarTok}[1]{\textcolor[rgb]{0.56,0.35,0.01}{\textbf{\textit{#1}}}}
\newcommand{\ConstantTok}[1]{\textcolor[rgb]{0.00,0.00,0.00}{#1}}
\newcommand{\ControlFlowTok}[1]{\textcolor[rgb]{0.13,0.29,0.53}{\textbf{#1}}}
\newcommand{\DataTypeTok}[1]{\textcolor[rgb]{0.13,0.29,0.53}{#1}}
\newcommand{\DecValTok}[1]{\textcolor[rgb]{0.00,0.00,0.81}{#1}}
\newcommand{\DocumentationTok}[1]{\textcolor[rgb]{0.56,0.35,0.01}{\textbf{\textit{#1}}}}
\newcommand{\ErrorTok}[1]{\textcolor[rgb]{0.64,0.00,0.00}{\textbf{#1}}}
\newcommand{\ExtensionTok}[1]{#1}
\newcommand{\FloatTok}[1]{\textcolor[rgb]{0.00,0.00,0.81}{#1}}
\newcommand{\FunctionTok}[1]{\textcolor[rgb]{0.00,0.00,0.00}{#1}}
\newcommand{\ImportTok}[1]{#1}
\newcommand{\InformationTok}[1]{\textcolor[rgb]{0.56,0.35,0.01}{\textbf{\textit{#1}}}}
\newcommand{\KeywordTok}[1]{\textcolor[rgb]{0.13,0.29,0.53}{\textbf{#1}}}
\newcommand{\NormalTok}[1]{#1}
\newcommand{\OperatorTok}[1]{\textcolor[rgb]{0.81,0.36,0.00}{\textbf{#1}}}
\newcommand{\OtherTok}[1]{\textcolor[rgb]{0.56,0.35,0.01}{#1}}
\newcommand{\PreprocessorTok}[1]{\textcolor[rgb]{0.56,0.35,0.01}{\textit{#1}}}
\newcommand{\RegionMarkerTok}[1]{#1}
\newcommand{\SpecialCharTok}[1]{\textcolor[rgb]{0.00,0.00,0.00}{#1}}
\newcommand{\SpecialStringTok}[1]{\textcolor[rgb]{0.31,0.60,0.02}{#1}}
\newcommand{\StringTok}[1]{\textcolor[rgb]{0.31,0.60,0.02}{#1}}
\newcommand{\VariableTok}[1]{\textcolor[rgb]{0.00,0.00,0.00}{#1}}
\newcommand{\VerbatimStringTok}[1]{\textcolor[rgb]{0.31,0.60,0.02}{#1}}
\newcommand{\WarningTok}[1]{\textcolor[rgb]{0.56,0.35,0.01}{\textbf{\textit{#1}}}}
\usepackage{graphicx,grffile}
\makeatletter
\def\maxwidth{\ifdim\Gin@nat@width>\linewidth\linewidth\else\Gin@nat@width\fi}
\def\maxheight{\ifdim\Gin@nat@height>\textheight\textheight\else\Gin@nat@height\fi}
\makeatother
% Scale images if necessary, so that they will not overflow the page
% margins by default, and it is still possible to overwrite the defaults
% using explicit options in \includegraphics[width, height, ...]{}
\setkeys{Gin}{width=\maxwidth,height=\maxheight,keepaspectratio}
% Set default figure placement to htbp
\makeatletter
\def\fps@figure{htbp}
\makeatother
\setlength{\emergencystretch}{3em} % prevent overfull lines
\providecommand{\tightlist}{%
  \setlength{\itemsep}{0pt}\setlength{\parskip}{0pt}}
\setcounter{secnumdepth}{-\maxdimen} % remove section numbering
\usepackage{booktabs}
\usepackage{longtable}
\usepackage{array}
\usepackage{multirow}
\usepackage{wrapfig}
\usepackage{float}
\usepackage{colortbl}
\usepackage{pdflscape}
\usepackage{tabu}
\usepackage{threeparttable}
\usepackage{threeparttablex}
\usepackage[normalem]{ulem}
\usepackage{makecell}
\usepackage{xcolor}

\title{DATA 605 - Discussion \#13}
\author{Zach Alexander}
\date{4/22/2020}

\begin{document}
\maketitle

\begin{center}\rule{0.5\linewidth}{0.5pt}\end{center}

\hypertarget{multiple-regression-analysis-of-covid-19-data-by-u.s.-county}{%
\subsubsection{Multiple Regression Analysis of COVID-19 Data by U.S.
County}\label{multiple-regression-analysis-of-covid-19-data-by-u.s.-county}}

\begin{center}\rule{0.5\linewidth}{0.5pt}\end{center}

Continuing my regression analysis from last week, this time around, I'll
add in additional terms, including a quadratic term, dichotomous term,
and a quadratic vs.~dichotomous interaction term to build out a multiple
regression.

From last week's analysis, it appears that population estimates were a
fairly good predictor of the number of confirmed cases of COVID-19 at a
county level. To make this more of a predictive model, we'll determine
whether population estimates, as well as other variables can accurately
predict the number of new confirmed cases for yesterday, April 21st,
based on counts from previous days. If the model fits the data well, we
could eventually see if we can apply this to future dates.

To do this, we'll first have to merge in some extra data. Fortunately, I
have tidy'd other census data in past semesters and can read in the data
file below that includes demographic, gender, age, education, income,
and employment data on a county level.

\begin{center}\rule{0.5\linewidth}{0.5pt}\end{center}

\hypertarget{step-1-downloading-the-data}{%
\subparagraph{Step 1: Downloading the
Data}\label{step-1-downloading-the-data}}

\begin{center}\rule{0.5\linewidth}{0.5pt}\end{center}

I decided to utilize the New York Times dataset that was made
open-source at the end of March. For more information about the
methodology of these confirmed case counts, you can find a
\href{https://github.com/nytimes/covid-19-data}{detailed explanation
here}.

Additionally, I'll be using U.S. Census data estimates as factors for my
regression model. This week, I decided to utilize population estimates
as my factor for my regression. I found a large dataset on the
\href{https://www.census.gov/data/datasets/time-series/demo/popest/2010s-counties-total.html\#par_textimage_70769902}{Census
website} with population estimates as recent as 2019.

And finally, as mentioned above, I downloaded the census data I compiled
from last semester as an additional file for this week.

I saved all three of these files to my github account and read them into
R:

\begin{Shaded}
\begin{Highlighting}[]
\NormalTok{county_covid <-}\StringTok{ }\KeywordTok{read.csv}\NormalTok{(}\StringTok{'https://raw.githubusercontent.com/zachalexander/data605_cuny/master/Homework13/Discussion13/us-counties_covid19_4222020.csv'}\NormalTok{, }\DataTypeTok{header =} \OtherTok{TRUE}\NormalTok{)}

\NormalTok{county_population <-}\StringTok{ }\KeywordTok{read.csv}\NormalTok{(}\StringTok{'https://raw.githubusercontent.com/zachalexander/data605_cuny/master/Homework12/Discussion12/county_population.csv'}\NormalTok{, }\DataTypeTok{header =} \OtherTok{TRUE}\NormalTok{)}

\NormalTok{education_data <-}\StringTok{ }\KeywordTok{read.csv}\NormalTok{(}\StringTok{'https://raw.githubusercontent.com/zachalexander/data_606_cunysps/master/Final_Project/education_data_fip.csv'}\NormalTok{, }\DataTypeTok{header =} \OtherTok{TRUE}\NormalTok{)}
\end{Highlighting}
\end{Shaded}

\begin{center}\rule{0.5\linewidth}{0.5pt}\end{center}

\hypertarget{step-2-tidying-the-data}{%
\subparagraph{Step 2: Tidying the Data}\label{step-2-tidying-the-data}}

\begin{center}\rule{0.5\linewidth}{0.5pt}\end{center}

With the data successfully loaded into R, I then had to do a bit of
tidying to ensure that my county-level data was accurate and reflected
the most up-to-date COVID-19 confirmed case counts.

First, I decided to tidy my COVID-19 data. Since the confirmed counts
are cumulative and broken out by each day since late January, I needed
to filter the dataset to only include the confirmed counts for April
14th, 2020.

\begin{Shaded}
\begin{Highlighting}[]
\NormalTok{covid_week <-}\StringTok{ }\NormalTok{county_covid }\OperatorTok\StringTok{ }
\StringTok{  }\KeywordTok{arrange}\NormalTok{(fips, date) }\OperatorTok\StringTok{ }
\StringTok{  }\KeywordTok{filter}\NormalTok{((}\KeywordTok{as.Date}\NormalTok{(date) }\OperatorTok{>=}\StringTok{ }\KeywordTok{as.Date}\NormalTok{(}\StringTok{'2020-04-12'}\NormalTok{)) }\OperatorTok{&}\StringTok{ }\NormalTok{(}\KeywordTok{as.Date}\NormalTok{(date) }\OperatorTok{<=}\StringTok{ }\KeywordTok{as.Date}\NormalTok{(}\StringTok{'2020-04-21'}\NormalTok{)))}

\NormalTok{covid_test <-}\StringTok{ }\NormalTok{covid_week }\OperatorTok\StringTok{ }
\StringTok{  }\KeywordTok{filter}\NormalTok{((}\KeywordTok{as.Date}\NormalTok{(date) }\OperatorTok{==}\StringTok{ }\KeywordTok{as.Date}\NormalTok{(}\StringTok{'2020-04-21'}\NormalTok{)))}

\NormalTok{covid_week_pivot <-}\StringTok{ }\NormalTok{covid_week }\OperatorTok\StringTok{ }
\StringTok{  }\KeywordTok{pivot_wider}\NormalTok{(}\DataTypeTok{id_cols =} \KeywordTok{c}\NormalTok{(date, fips, state, county), }\DataTypeTok{names_from =}\NormalTok{ date, }\DataTypeTok{values_from =}\NormalTok{ cases)}

\NormalTok{diff_list <-}\StringTok{ }\KeywordTok{c}\NormalTok{(}\StringTok{"413_diff"}\NormalTok{, }\StringTok{"414_diff"}\NormalTok{, }\StringTok{"415_diff"}\NormalTok{, }\StringTok{"416_diff"}\NormalTok{, }\StringTok{"417_diff"}\NormalTok{, }\StringTok{"418_diff"}\NormalTok{, }\StringTok{"419_diff"}\NormalTok{, }\StringTok{"420_diff"}\NormalTok{)}
\NormalTok{covid_week_pivot <-}\StringTok{ }\KeywordTok{cbind}\NormalTok{(covid_week_pivot, }\KeywordTok{setNames}\NormalTok{( }\KeywordTok{lapply}\NormalTok{(diff_list, }
                                         \ControlFlowTok{function}\NormalTok{(x) }\DataTypeTok{x=}\OtherTok{NA}\NormalTok{), diff_list) )}

\ControlFlowTok{for}\NormalTok{(i }\ControlFlowTok{in} \DecValTok{1}\OperatorTok{:}\KeywordTok{length}\NormalTok{(covid_week_pivot}\OperatorTok{$}\NormalTok{fips))\{}
  \ControlFlowTok{for}\NormalTok{(j }\ControlFlowTok{in} \DecValTok{1}\OperatorTok{:}\DecValTok{8}\NormalTok{)\{}
\NormalTok{  covid_week_pivot[i,(j}\OperatorTok{+}\DecValTok{12}\NormalTok{)] <-}\StringTok{ }\NormalTok{covid_week_pivot[i, (j}\OperatorTok{+}\DecValTok{4}\NormalTok{)] }\OperatorTok{-}\StringTok{ }\NormalTok{covid_week_pivot[i, (j}\OperatorTok{+}\DecValTok{3}\NormalTok{)]}
\NormalTok{  \}}
\NormalTok{\}}

\NormalTok{covid_week_fnl <-}\StringTok{ }\NormalTok{covid_week_pivot }\OperatorTok\StringTok{ }
\StringTok{  }\KeywordTok{select}\NormalTok{(fips, state, county, }\StringTok{`}\DataTypeTok{413_diff}\StringTok{`}\NormalTok{, }\StringTok{`}\DataTypeTok{414_diff}\StringTok{`}\NormalTok{, }\StringTok{`}\DataTypeTok{415_diff}\StringTok{`}\NormalTok{, }\StringTok{`}\DataTypeTok{416_diff}\StringTok{`}\NormalTok{, }\StringTok{`}\DataTypeTok{417_diff}\StringTok{`}\NormalTok{, }\StringTok{`}\DataTypeTok{418_diff}\StringTok{`}\NormalTok{, }\StringTok{`}\DataTypeTok{419_diff}\StringTok{`}\NormalTok{, }\StringTok{`}\DataTypeTok{420_diff}\StringTok{`}\NormalTok{, }\StringTok{`}\DataTypeTok{2020-04-20}\StringTok{`}\NormalTok{)}


\NormalTok{covid_week_fnl <-}\StringTok{ }\KeywordTok{mutate}\NormalTok{(covid_week_fnl, }\DataTypeTok{week_avg =} \KeywordTok{rowMeans}\NormalTok{(covid_week_fnl[, }\DecValTok{4}\OperatorTok{:}\DecValTok{11}\NormalTok{], }\DataTypeTok{na.rm =} \OtherTok{TRUE}\NormalTok{))}
\end{Highlighting}
\end{Shaded}

Next, I noticed that the New York Times groups the confirmed counts for
New York City into one row, instead of breaking it into the five
counties that comprise of ``New York City'' (Queens, Kings, New York,
Bronx and Richmond). Therefore, I decided to use the FIPS code
associated with New York County (36061) as my identifier when I
eventually merge the population data into the confirmed case data. I
then adjusted the population estimate to reflect the population of all
five counties instead of just New York County (seen later).

\begin{Shaded}
\begin{Highlighting}[]
\NormalTok{covid_week_fnl <-}\StringTok{ }\KeywordTok{within}\NormalTok{(covid_week_fnl, \{}
\NormalTok{    f <-}\StringTok{ }\NormalTok{county }\OperatorTok{==}\StringTok{ 'New York City'}
\NormalTok{    fips[f] <-}\StringTok{ '36061'}
\NormalTok{\}) }

\NormalTok{covid_test <-}\StringTok{ }\KeywordTok{within}\NormalTok{(covid_test, \{}
\NormalTok{    f <-}\StringTok{ }\NormalTok{county }\OperatorTok{==}\StringTok{ 'New York City'}
\NormalTok{    fips[f] <-}\StringTok{ '36061'}
\NormalTok{\}) }

\NormalTok{covid_test <-}\StringTok{ }\NormalTok{covid_test }\OperatorTok\StringTok{ }
\StringTok{  }\KeywordTok{select}\NormalTok{(fips, state, county, cases)}
\end{Highlighting}
\end{Shaded}

For the merge, I also thought it would be helpful to make the county
names consistent across both datasets.

\begin{Shaded}
\begin{Highlighting}[]
\NormalTok{covid_week_fnl}\OperatorTok{$}\NormalTok{county <-}\StringTok{ }\KeywordTok{paste}\NormalTok{(covid_week_fnl}\OperatorTok{$}\NormalTok{county, }\StringTok{'County'}\NormalTok{)}
\end{Highlighting}
\end{Shaded}

With the COVID-19 data file ready for the merge, I then turned my
attention to my population data file. In order to use county FIPS codes
as my identifier in both datasets, I had to generate the FIPS codes in
the population file.

\begin{Shaded}
\begin{Highlighting}[]
\NormalTok{county_population <-}\StringTok{ }\NormalTok{county_population }\OperatorTok\StringTok{ }
\StringTok{  }\KeywordTok{select}\NormalTok{(STATE, COUNTY, STNAME, CTYNAME, POPESTIMATE2019)}

\NormalTok{county_population}\OperatorTok{$}\NormalTok{fips <-}\StringTok{ }\OtherTok{NA}
\ControlFlowTok{for}\NormalTok{ (i }\ControlFlowTok{in} \DecValTok{1}\OperatorTok{:}\KeywordTok{length}\NormalTok{(county_population}\OperatorTok{$}\NormalTok{STATE)) \{}
  \ControlFlowTok{if}\NormalTok{(county_population}\OperatorTok{$}\NormalTok{COUNTY[i] }\OperatorTok{<}\StringTok{ }\DecValTok{10}\NormalTok{) \{}
\NormalTok{    county_population}\OperatorTok{$}\NormalTok{fips[i] <-}\StringTok{ }\KeywordTok{paste0}\NormalTok{(county_population}\OperatorTok{$}\NormalTok{STATE[i], }\StringTok{'00'}\NormalTok{, county_population}\OperatorTok{$}\NormalTok{COUNTY[i])}
\NormalTok{  \}}
  \ControlFlowTok{if}\NormalTok{(county_population}\OperatorTok{$}\NormalTok{COUNTY[i] }\OperatorTok{<}\StringTok{ }\DecValTok{100} \OperatorTok{&}\StringTok{ }\NormalTok{county_population}\OperatorTok{$}\NormalTok{COUNTY[i] }\OperatorTok{>=}\StringTok{ }\DecValTok{10}\NormalTok{) \{}
\NormalTok{    county_population}\OperatorTok{$}\NormalTok{fips[i] <-}\StringTok{ }\KeywordTok{paste0}\NormalTok{(county_population}\OperatorTok{$}\NormalTok{STATE[i], }\StringTok{'0'}\NormalTok{, county_population}\OperatorTok{$}\NormalTok{COUNTY[i])}
\NormalTok{  \}}
  \ControlFlowTok{if}\NormalTok{(county_population}\OperatorTok{$}\NormalTok{COUNTY[i] }\OperatorTok{>=}\DecValTok{100}\NormalTok{) \{}
\NormalTok{    county_population}\OperatorTok{$}\NormalTok{fips[i] <-}\StringTok{ }\KeywordTok{paste0}\NormalTok{(county_population}\OperatorTok{$}\NormalTok{STATE[i], county_population}\OperatorTok{$}\NormalTok{COUNTY[i])}
\NormalTok{  \}}
\NormalTok{\}}

\NormalTok{county_population <-}\StringTok{ }\NormalTok{county_population }\OperatorTok\StringTok{ }
\StringTok{  }\KeywordTok{select}\NormalTok{(fips, CTYNAME, STNAME, POPESTIMATE2019)}
\KeywordTok{names}\NormalTok{(county_population) <-}\StringTok{ }\KeywordTok{c}\NormalTok{(}\StringTok{'fips'}\NormalTok{, }\StringTok{'county'}\NormalTok{, }\StringTok{'state'}\NormalTok{, }\StringTok{'pop_estimate'}\NormalTok{)}
\end{Highlighting}
\end{Shaded}

I also noticed that this file had overall state population estimates, so
before merging, I made sure to filter these out.

\begin{Shaded}
\begin{Highlighting}[]
\NormalTok{county_population <-}\StringTok{ }\NormalTok{county_population }\OperatorTok\StringTok{ }
\StringTok{  }\KeywordTok{filter}\NormalTok{(}\KeywordTok{grepl}\NormalTok{(}\StringTok{"County"}\NormalTok{,county))}
\end{Highlighting}
\end{Shaded}

With both data files ready to go, I then merged the population estimates
data into the COVID-19 data file and created one final dataframe. I also
updated the population count for New York City to ensure that it didn't
just account for the population in New York County, but also the four
other counties in the metropolitan area.

\begin{Shaded}
\begin{Highlighting}[]
\NormalTok{fnl <-}\StringTok{ }\KeywordTok{merge}\NormalTok{(covid_week_fnl, county_population, }\DataTypeTok{by=}\StringTok{'fips'}\NormalTok{)}

\NormalTok{fnl <-}\StringTok{ }\NormalTok{fnl }\OperatorTok\StringTok{ }
\StringTok{  }\KeywordTok{select}\NormalTok{(fips, county.x, state.x, }\StringTok{`}\DataTypeTok{2020-04-20}\StringTok{`}\NormalTok{, week_avg, pop_estimate)}

\KeywordTok{names}\NormalTok{(fnl) <-}\StringTok{ }\KeywordTok{c}\NormalTok{(}\StringTok{'fips'}\NormalTok{, }\StringTok{'county'}\NormalTok{, }\StringTok{'state'}\NormalTok{, }\StringTok{'April20cases'}\NormalTok{, }\StringTok{'week_avg'}\NormalTok{, }\StringTok{'pop_estimate'}\NormalTok{)}

\NormalTok{fnl}\OperatorTok{$}\NormalTok{pop_estimate[fnl}\OperatorTok{$}\NormalTok{county }\OperatorTok{==}\StringTok{ 'New York City County'}\NormalTok{] <-}\StringTok{ }\DecValTok{8398748}
\NormalTok{fnl}\OperatorTok{$}\NormalTok{county[fnl}\OperatorTok{$}\NormalTok{county }\OperatorTok{==}\StringTok{ 'New York City County'}\NormalTok{] <-}\StringTok{ 'New York City'}

\NormalTok{fnl <-}\StringTok{ }\NormalTok{fnl }\OperatorTok\StringTok{ }
\StringTok{  }\KeywordTok{arrange}\NormalTok{(}\KeywordTok{desc}\NormalTok{(week_avg))}
\end{Highlighting}
\end{Shaded}

Here is a look at my final dataframe, ready to start my regression
analysis:

\begin{Shaded}
\begin{Highlighting}[]
\KeywordTok{kable}\NormalTok{(}\KeywordTok{head}\NormalTok{(fnl, }\DataTypeTok{n=}\NormalTok{15L)) }\OperatorTok
\StringTok{  }\KeywordTok{kable_styling}\NormalTok{(}\DataTypeTok{bootstrap_options =} \KeywordTok{c}\NormalTok{(}\StringTok{"striped"}\NormalTok{, }\StringTok{"hover"}\NormalTok{))}
\end{Highlighting}
\end{Shaded}

\begin{table}[H]
\centering
\begin{tabular}{l|l|l|r|r|r}
\hline
fips & county & state & April20cases & week\_avg & pop\_estimate\\
\hline
36061 & New York City & New York & 136816 & 4293.1429 & 8398748\\
\hline
17031 & Cook County & Illinois & 22101 & 946.7143 & 5150233\\
\hline
36059 & Nassau County & New York & 30677 & 902.7143 & 1356924\\
\hline
36103 & Suffolk County & New York & 27662 & 859.8571 & 1476601\\
\hline
36119 & Westchester County & New York & 24306 & 645.8571 & 967506\\
\hline
6037 & Los Angeles County & California & 13816 & 628.0000 & 10039107\\
\hline
34039 & Union County & New Jersey & 9972 & 476.5714 & 556341\\
\hline
34017 & Hudson County & New Jersey & 11150 & 467.2857 & 672391\\
\hline
25017 & Middlesex County & Massachusetts & 9253 & 467.1429 & 1611699\\
\hline
34013 & Essex County & New Jersey & 10729 & 442.1429 & 798975\\
\hline
34003 & Bergen County & New Jersey & 13011 & 417.0000 & 932202\\
\hline
42101 & Philadelphia County & Pennsylvania & 9553 & 391.8571 & 1584064\\
\hline
25025 & Suffolk County & Massachusetts & 8314 & 390.7143 & 803907\\
\hline
34031 & Passaic County & New Jersey & 8479 & 361.2857 & 501826\\
\hline
34023 & Middlesex County & New Jersey & 8346 & 337.0000 & 825062\\
\hline
\end{tabular}
\end{table}

We'll then merge in a few more terms from the other census dataset:

\begin{Shaded}
\begin{Highlighting}[]
\NormalTok{fnl <-}\StringTok{ }\KeywordTok{merge}\NormalTok{(fnl, education_data, }\DataTypeTok{by=}\StringTok{'fips'}\NormalTok{)}

\NormalTok{fnl <-}\StringTok{ }\NormalTok{fnl }\OperatorTok\StringTok{ }
\StringTok{  }\KeywordTok{select}\NormalTok{(fips, county.x, state.x, April20cases, week_avg, pop_estimate, median_hh_inc, age65andolder_pct, nonwhite_pct, lesscollege_pct)}
\end{Highlighting}
\end{Shaded}

\begin{Shaded}
\begin{Highlighting}[]
\KeywordTok{par}\NormalTok{(}\DataTypeTok{mfrow=}\KeywordTok{c}\NormalTok{(}\DecValTok{3}\NormalTok{,}\DecValTok{3}\NormalTok{))}
\KeywordTok{hist}\NormalTok{(fnl}\OperatorTok{$}\NormalTok{April20cases)}
\KeywordTok{hist}\NormalTok{(fnl}\OperatorTok{$}\NormalTok{week_avg)}
\KeywordTok{hist}\NormalTok{(fnl}\OperatorTok{$}\NormalTok{pop_estimate)}
\KeywordTok{hist}\NormalTok{(fnl}\OperatorTok{$}\NormalTok{median_hh_inc)}
\KeywordTok{hist}\NormalTok{(fnl}\OperatorTok{$}\NormalTok{age65andolder_pct)}
\KeywordTok{hist}\NormalTok{(fnl}\OperatorTok{$}\NormalTok{nonwhite_pct)}
\KeywordTok{hist}\NormalTok{(fnl}\OperatorTok{$}\NormalTok{lesscollege_pct)}
\end{Highlighting}
\end{Shaded}

\includegraphics{ZAlexander_discuss13_files/figure-latex/unnamed-chunk-9-1.pdf}

\hypertarget{dichotomous-variable}{%
\subparagraph{Dichotomous Variable}\label{dichotomous-variable}}

\begin{Shaded}
\begin{Highlighting}[]
\KeywordTok{mean}\NormalTok{(fnl}\OperatorTok{$}\NormalTok{median_hh_inc)}
\end{Highlighting}
\end{Shaded}

\begin{verbatim}
## [1] 48349.44
\end{verbatim}

It looks like we could create a dichotomous term from our median
household income variable. By doing this, we'll create a new variable
called \texttt{median\_hh\_inc\_dichot}, which will be a value of 1 if
the median household income is greater than \$48,300 and 0 if the median
household income is less than this value for each county:

\begin{Shaded}
\begin{Highlighting}[]
\ControlFlowTok{for}\NormalTok{ (i }\ControlFlowTok{in} \DecValTok{1}\OperatorTok{:}\KeywordTok{length}\NormalTok{(fnl}\OperatorTok{$}\NormalTok{fips)) \{}
  \ControlFlowTok{if}\NormalTok{(fnl}\OperatorTok{$}\NormalTok{median_hh_inc[i] }\OperatorTok{>}\StringTok{ }\DecValTok{43000}\NormalTok{)\{}
\NormalTok{    fnl}\OperatorTok{$}\NormalTok{median_hh_inc_dichot[i] <-}\StringTok{ }\DecValTok{1}
\NormalTok{  \} }\ControlFlowTok{else}\NormalTok{ \{}
\NormalTok{    fnl}\OperatorTok{$}\NormalTok{median_hh_inc_dichot[i] <-}\StringTok{ }\DecValTok{0}
\NormalTok{  \}}
\NormalTok{\}}
\end{Highlighting}
\end{Shaded}

And here's the split between the two factors of this new dichotomous
variable:

\begin{Shaded}
\begin{Highlighting}[]
\KeywordTok{table}\NormalTok{(fnl}\OperatorTok{$}\NormalTok{median_hh_inc_dichot)}
\end{Highlighting}
\end{Shaded}

\begin{verbatim}
## 
##    0    1 
##  953 1690
\end{verbatim}

\hypertarget{quadratic-variable}{%
\subparagraph{Quadratic Variable}\label{quadratic-variable}}

It looks like from our initial plots of histograms above, we can
transform our \texttt{nonwhite\_pct} variable by taking the square root,
to make it more of a normal distribution. By doing this, we can now see
that the distribution is approaching normal:

\begin{Shaded}
\begin{Highlighting}[]
\KeywordTok{par}\NormalTok{(}\DataTypeTok{mfrow=}\KeywordTok{c}\NormalTok{(}\DecValTok{3}\NormalTok{,}\DecValTok{3}\NormalTok{))}
\KeywordTok{hist}\NormalTok{(fnl}\OperatorTok{$}\NormalTok{April20cases)}
\KeywordTok{hist}\NormalTok{(fnl}\OperatorTok{$}\NormalTok{week_avg)}
\KeywordTok{hist}\NormalTok{(}\DecValTok{1} \OperatorTok{/}\StringTok{ }\KeywordTok{log}\NormalTok{(fnl}\OperatorTok{$}\NormalTok{pop_estimate))}
\KeywordTok{hist}\NormalTok{(fnl}\OperatorTok{$}\NormalTok{median_hh_inc)}
\KeywordTok{hist}\NormalTok{(fnl}\OperatorTok{$}\NormalTok{age65andolder_pct)}
\KeywordTok{hist}\NormalTok{(}\KeywordTok{sqrt}\NormalTok{(fnl}\OperatorTok{$}\NormalTok{nonwhite_pct))}
\KeywordTok{hist}\NormalTok{(fnl}\OperatorTok{$}\NormalTok{lesscollege_pct)}

\NormalTok{fnl}\OperatorTok{$}\NormalTok{pop_estimate <-}\StringTok{ }\DecValTok{1} \OperatorTok{/}\StringTok{ }\KeywordTok{log}\NormalTok{(fnl}\OperatorTok{$}\NormalTok{pop_estimate)}
\NormalTok{fnl}\OperatorTok{$}\NormalTok{nonwhite_pct <-}\StringTok{ }\KeywordTok{sqrt}\NormalTok{(fnl}\OperatorTok{$}\NormalTok{nonwhite_pct)}

\KeywordTok{par}\NormalTok{(}\DataTypeTok{mfrow=}\KeywordTok{c}\NormalTok{(}\DecValTok{3}\NormalTok{,}\DecValTok{3}\NormalTok{))}
\end{Highlighting}
\end{Shaded}

\includegraphics{ZAlexander_discuss13_files/figure-latex/unnamed-chunk-13-1.pdf}

\begin{Shaded}
\begin{Highlighting}[]
\KeywordTok{hist}\NormalTok{(fnl}\OperatorTok{$}\NormalTok{April20cases)}
\KeywordTok{hist}\NormalTok{(fnl}\OperatorTok{$}\NormalTok{week_avg)}
\KeywordTok{hist}\NormalTok{(fnl}\OperatorTok{$}\NormalTok{pop_estimate)}
\KeywordTok{hist}\NormalTok{(fnl}\OperatorTok{$}\NormalTok{median_hh_inc)}
\KeywordTok{hist}\NormalTok{(fnl}\OperatorTok{$}\NormalTok{age65andolder_pct)}
\KeywordTok{hist}\NormalTok{(fnl}\OperatorTok{$}\NormalTok{nonwhite_pct)}
\KeywordTok{hist}\NormalTok{(fnl}\OperatorTok{$}\NormalTok{lesscollege_pct)}
\end{Highlighting}
\end{Shaded}

\includegraphics{ZAlexander_discuss13_files/figure-latex/unnamed-chunk-13-2.pdf}

Additionally, I did a log transformation on the population estimate term
to make it more normalized as well.

\hypertarget{quadratic-and-dichotomous-interaction-term}{%
\subparagraph{Quadratic and Dichotomous Interaction
Term}\label{quadratic-and-dichotomous-interaction-term}}

When we run our linear regression model, we'll be sure to create an
interaction term between our non-white percentage variable (quadratic
transformation) and our dichotomous median household income variable.

\begin{center}\rule{0.5\linewidth}{0.5pt}\end{center}

\hypertarget{initial-analysis}{%
\subparagraph{Initial Analysis}\label{initial-analysis}}

\begin{center}\rule{0.5\linewidth}{0.5pt}\end{center}

With our variables and dataset ready to go, we will now take a quick
look at interactions between variables. We can do this using the
\texttt{pairs()} function:

\begin{Shaded}
\begin{Highlighting}[]
\NormalTok{fnl_pairs <-}\StringTok{ }\NormalTok{fnl }\OperatorTok\StringTok{ }
\StringTok{  }\KeywordTok{select}\NormalTok{(April20cases, week_avg, pop_estimate, median_hh_inc, median_hh_inc_dichot, age65andolder_pct,nonwhite_pct, lesscollege_pct)}
\KeywordTok{pairs}\NormalTok{(fnl_pairs)}
\end{Highlighting}
\end{Shaded}

\includegraphics{ZAlexander_discuss13_files/figure-latex/unnamed-chunk-14-1.pdf}

If you look closely, we can see that there seems to be a somewhat strong
interaction between median household income and counties with a larger
percentage of their population with less than a college degree -- this
make sense intuitively. Other interactions are not as obvious, so we'll
see how this fairs in our regression model.

\begin{center}\rule{0.5\linewidth}{0.5pt}\end{center}

\hypertarget{building-the-one-factor-linear-regression}{%
\subparagraph{Building the One-Factor Linear
Regression}\label{building-the-one-factor-linear-regression}}

\begin{center}\rule{0.5\linewidth}{0.5pt}\end{center}

With our factors ready to go, we can create a multiple linear regression
model.

\begin{Shaded}
\begin{Highlighting}[]
\NormalTok{covid_lm <-}\StringTok{ }\KeywordTok{lm}\NormalTok{(fnl}\OperatorTok{$}\NormalTok{April20cases }\OperatorTok{~}\StringTok{ }\NormalTok{fnl}\OperatorTok{$}\NormalTok{April20cases }\OperatorTok{+}\StringTok{ }\NormalTok{fnl}\OperatorTok{$}\NormalTok{week_avg }\OperatorTok{+}\StringTok{ }\NormalTok{fnl}\OperatorTok{$}\NormalTok{pop_estimate }\OperatorTok{+}\StringTok{ }\NormalTok{fnl}\OperatorTok{$}\NormalTok{median_hh_inc }\OperatorTok{+}\StringTok{ }\NormalTok{fnl}\OperatorTok{$}\NormalTok{median_hh_inc_dichot }\OperatorTok{+}\StringTok{ }\NormalTok{fnl}\OperatorTok{$}\NormalTok{age65andolder_pct }\OperatorTok{+}\StringTok{ }\NormalTok{fnl}\OperatorTok{$}\NormalTok{nonwhite_pct }\OperatorTok{+}\StringTok{ }\NormalTok{fnl}\OperatorTok{$}\NormalTok{lesscollege_pct }\OperatorTok{+}\StringTok{ }\NormalTok{fnl}\OperatorTok{$}\NormalTok{median_hh_inc_dichot}\OperatorTok{:}\NormalTok{fnl}\OperatorTok{$}\NormalTok{nonwhite_pct)}
\end{Highlighting}
\end{Shaded}

\begin{verbatim}
## Warning in model.matrix.default(mt, mf, contrasts): the response appeared on the
## right-hand side and was dropped
\end{verbatim}

\begin{verbatim}
## Warning in model.matrix.default(mt, mf, contrasts): problem with term 1 in
## model.matrix: no columns are assigned
\end{verbatim}

\begin{Shaded}
\begin{Highlighting}[]
\NormalTok{covid_lm}
\end{Highlighting}
\end{Shaded}

\begin{verbatim}
## 
## Call:
## lm(formula = fnl$April20cases ~ fnl$April20cases + fnl$week_avg + 
##     fnl$pop_estimate + fnl$median_hh_inc + fnl$median_hh_inc_dichot + 
##     fnl$age65andolder_pct + fnl$nonwhite_pct + fnl$lesscollege_pct + 
##     fnl$median_hh_inc_dichot:fnl$nonwhite_pct)
## 
## Coefficients:
##                               (Intercept)  
##                                -2.861e+02  
##                              fnl$week_avg  
##                                 3.079e+01  
##                          fnl$pop_estimate  
##                                 3.396e+03  
##                         fnl$median_hh_inc  
##                                -3.765e-05  
##                  fnl$median_hh_inc_dichot  
##                                 7.391e+01  
##                     fnl$age65andolder_pct  
##                                 5.995e-01  
##                          fnl$nonwhite_pct  
##                                -3.333e+00  
##                       fnl$lesscollege_pct  
##                                -7.393e-01  
## fnl$median_hh_inc_dichot:fnl$nonwhite_pct  
##                                -2.538e+01
\end{verbatim}

Above, we can see the intercept and slope of our linear regression
(8.962 and 0.5914 respectively). To get a more detailed outlook of the
performance of our model, we can use the \texttt{summary()} function in
R:

\begin{Shaded}
\begin{Highlighting}[]
\KeywordTok{summary}\NormalTok{(covid_lm)}
\end{Highlighting}
\end{Shaded}

\begin{verbatim}
## 
## Call:
## lm(formula = fnl$April20cases ~ fnl$April20cases + fnl$week_avg + 
##     fnl$pop_estimate + fnl$median_hh_inc + fnl$median_hh_inc_dichot + 
##     fnl$age65andolder_pct + fnl$nonwhite_pct + fnl$lesscollege_pct + 
##     fnl$median_hh_inc_dichot:fnl$nonwhite_pct)
## 
## Residuals:
##     Min      1Q  Median      3Q     Max 
## -6797.7   -19.8    14.6    59.5  4860.8 
## 
## Coefficients:
##                                             Estimate Std. Error t value
## (Intercept)                               -2.861e+02  1.547e+02  -1.849
## fnl$week_avg                               3.079e+01  8.542e-02 360.452
## fnl$pop_estimate                           3.396e+03  8.295e+02   4.094
## fnl$median_hh_inc                         -3.765e-05  1.120e-03  -0.034
## fnl$median_hh_inc_dichot                   7.391e+01  4.244e+01   1.742
## fnl$age65andolder_pct                      5.995e-01  2.277e+00   0.263
## fnl$nonwhite_pct                          -3.333e+00  6.278e+00  -0.531
## fnl$lesscollege_pct                       -7.393e-01  1.291e+00  -0.573
## fnl$median_hh_inc_dichot:fnl$nonwhite_pct -2.538e+01  8.695e+00  -2.919
##                                           Pr(>|t|)    
## (Intercept)                                0.06457 .  
## fnl$week_avg                               < 2e-16 ***
## fnl$pop_estimate                          4.37e-05 ***
## fnl$median_hh_inc                          0.97318    
## fnl$median_hh_inc_dichot                   0.08170 .  
## fnl$age65andolder_pct                      0.79234    
## fnl$nonwhite_pct                           0.59548    
## fnl$lesscollege_pct                        0.56700    
## fnl$median_hh_inc_dichot:fnl$nonwhite_pct  0.00354 ** 
## ---
## Signif. codes:  0 '***' 0.001 '**' 0.01 '*' 0.05 '.' 0.1 ' ' 1
## 
## Residual standard error: 408.4 on 2615 degrees of freedom
##   (19 observations deleted due to missingness)
## Multiple R-squared:  0.9814, Adjusted R-squared:  0.9814 
## F-statistic: 1.727e+04 on 8 and 2615 DF,  p-value: < 2.2e-16
\end{verbatim}

\hypertarget{backward-elimination}{%
\subparagraph{Backward Elimination}\label{backward-elimination}}

It looks like our \texttt{age65andolder\_pct} variable has the highest
p-value, and others, including \texttt{median\_hh\_inc} and
\texttt{lesscollege\_pct} will likely be removed when we do backward
elimination because they also exhibit pretty high p-values. We can
double check this by running the stepwise process:

\begin{Shaded}
\begin{Highlighting}[]
\NormalTok{covid_lm <-}\StringTok{ }\KeywordTok{step}\NormalTok{(covid_lm, }\DataTypeTok{direction =} \StringTok{'backward'}\NormalTok{, }\DataTypeTok{trace =} \OtherTok{FALSE}\NormalTok{)}
\end{Highlighting}
\end{Shaded}

\begin{verbatim}
## Warning in model.matrix.default(object, data = structure(list(`fnl$April20cases`
## = c(430L, : the response appeared on the right-hand side and was dropped
\end{verbatim}

\begin{verbatim}
## Warning in model.matrix.default(object, data = structure(list(`fnl$April20cases`
## = c(430L, : problem with term 1 in model.matrix: no columns are assigned
\end{verbatim}

\begin{Shaded}
\begin{Highlighting}[]
\KeywordTok{summary}\NormalTok{(covid_lm)}
\end{Highlighting}
\end{Shaded}

\begin{verbatim}
## 
## Call:
## lm(formula = fnl$April20cases ~ fnl$week_avg + fnl$pop_estimate + 
##     fnl$median_hh_inc_dichot + fnl$nonwhite_pct + fnl$median_hh_inc_dichot:fnl$nonwhite_pct)
## 
## Residuals:
##     Min      1Q  Median      3Q     Max 
## -6800.5   -19.7    13.7    59.7  4850.7 
## 
## Coefficients:
##                                             Estimate Std. Error t value
## (Intercept)                               -324.54516   81.98072  -3.959
## fnl$week_avg                                30.79764    0.08459 364.070
## fnl$pop_estimate                          3271.69836  740.05432   4.421
## fnl$median_hh_inc_dichot                    77.07769   41.10527   1.875
## fnl$nonwhite_pct                            -3.77270    5.98143  -0.631
## fnl$median_hh_inc_dichot:fnl$nonwhite_pct  -25.23837    8.63175  -2.924
##                                           Pr(>|t|)    
## (Intercept)                               7.73e-05 ***
## fnl$week_avg                               < 2e-16 ***
## fnl$pop_estimate                          1.02e-05 ***
## fnl$median_hh_inc_dichot                   0.06089 .  
## fnl$nonwhite_pct                           0.52827    
## fnl$median_hh_inc_dichot:fnl$nonwhite_pct  0.00349 ** 
## ---
## Signif. codes:  0 '***' 0.001 '**' 0.01 '*' 0.05 '.' 0.1 ' ' 1
## 
## Residual standard error: 408.2 on 2618 degrees of freedom
##   (19 observations deleted due to missingness)
## Multiple R-squared:  0.9814, Adjusted R-squared:  0.9814 
## F-statistic: 2.767e+04 on 5 and 2618 DF,  p-value: < 2.2e-16
\end{verbatim}

\begin{center}\rule{0.5\linewidth}{0.5pt}\end{center}

\hypertarget{evaluating-the-model-and-residual-analysis}{%
\subparagraph{Evaluating the Model and Residual
Analysis}\label{evaluating-the-model-and-residual-analysis}}

\begin{center}\rule{0.5\linewidth}{0.5pt}\end{center}

After running the summary above, we can see a few things:

\begin{itemize}
\item
  The median residual value is around zero (at 14 cases), which is a
  good sign.
\item
  Additionally, the minimum and maximum values of the residuals are
  roughly around the same, albeit leaning a bit more on the minimum side
  of at -7192 and 4883 respectively.
\item
  Our Multiple R-squared value is 0.9825, which indicates that these
  terms by county account for about 98.25\% of the variability in the
  number of confirmed COVID-19 cases by county for April 20th, 2020.
  This seems to be a pretty good result for our regression model.
\end{itemize}

Although this seems to be a good sign, we need to check our residuals to
see if this qualifies as a suitable regression model.

When plotting residuals (below), we can see that residuals are not
uniformly scattered around zero:

\begin{Shaded}
\begin{Highlighting}[]
\KeywordTok{plot}\NormalTok{(}\KeywordTok{fitted}\NormalTok{(covid_lm),}\KeywordTok{resid}\NormalTok{(covid_lm))}
\end{Highlighting}
\end{Shaded}

\includegraphics{ZAlexander_discuss13_files/figure-latex/unnamed-chunk-18-1.pdf}

I'm not sure if this is due to some very large outliers and skewed cases
data, but we can also see that there are a fair amount of outliers
towards both ends of the q-q plot. This can be visualized in a
quantile-versus-quantile (Q-Q) plot (see below):

\begin{Shaded}
\begin{Highlighting}[]
\KeywordTok{qqnorm}\NormalTok{(}\KeywordTok{resid}\NormalTok{(covid_lm))}
\KeywordTok{qqline}\NormalTok{(}\KeywordTok{resid}\NormalTok{(covid_lm))}
\end{Highlighting}
\end{Shaded}

\includegraphics{ZAlexander_discuss13_files/figure-latex/unnamed-chunk-19-1.pdf}

\begin{Shaded}
\begin{Highlighting}[]
\KeywordTok{hist}\NormalTok{(}\KeywordTok{resid}\NormalTok{(covid_lm))}
\end{Highlighting}
\end{Shaded}

\includegraphics{ZAlexander_discuss13_files/figure-latex/unnamed-chunk-19-2.pdf}

\begin{center}\rule{0.5\linewidth}{0.5pt}\end{center}

\hypertarget{was-the-linear-model-appropriate}{%
\subparagraph{Was the Linear Model
Appropriate?}\label{was-the-linear-model-appropriate}}

\begin{center}\rule{0.5\linewidth}{0.5pt}\end{center}

From this residual analysis, we can conclude that a linear model here
may not be very appropriate, or accurate for areas with larger numbers
of confirmed cases (i.e.~counties with larger populations). However,
this may be effective to predict cases for counties with smaller
populations. At this point, I think I would not call this linear model
appropriate, however, further manipulation of the initial dataset to
exclude counties with very large populations may make this a more
effective model.

\hypertarget{predictions-for-april-21st-2020}{%
\subparagraph{Predictions for April 21st,
2020}\label{predictions-for-april-21st-2020}}

From the t-test below, it does look like a 95\% confidence interval
includes zero. However, it isn't a very tight range between the lower
and upper bounds, indicating that the predictions for confirmed cases
could be drastically off, especially for smaller populations. This
confirms that it would be beneficial to adjust the original dataset to
build out a more productive and efficient model.

\begin{Shaded}
\begin{Highlighting}[]
\NormalTok{predicted <-}\StringTok{ }\KeywordTok{predict}\NormalTok{(covid_lm, covid_test)}
\end{Highlighting}
\end{Shaded}

\begin{verbatim}
## Warning: 'newdata' had 2783 rows but variables found have 2643 rows
\end{verbatim}

\begin{Shaded}
\begin{Highlighting}[]
\NormalTok{delta <-}\StringTok{ }\NormalTok{predicted }\OperatorTok{-}\StringTok{ }\NormalTok{covid_test}\OperatorTok{$}\NormalTok{cases}
\end{Highlighting}
\end{Shaded}

\begin{verbatim}
## Warning in predicted - covid_test$cases: longer object length is not a multiple
## of shorter object length
\end{verbatim}

\begin{Shaded}
\begin{Highlighting}[]
\KeywordTok{t.test}\NormalTok{(delta, }\DataTypeTok{conf.level =} \FloatTok{0.95}\NormalTok{)}
\end{Highlighting}
\end{Shaded}

\begin{verbatim}
## 
##  One Sample t-test
## 
## data:  delta
## t = -0.16414, df = 2765, p-value = 0.8696
## alternative hypothesis: true mean is not equal to 0
## 95 percent confidence interval:
##  -167.8447  141.9148
## sample estimates:
## mean of x 
## -12.96496
\end{verbatim}

\begin{Shaded}
\begin{Highlighting}[]
\KeywordTok{plot}\NormalTok{(delta)}
\end{Highlighting}
\end{Shaded}

\includegraphics{ZAlexander_discuss13_files/figure-latex/unnamed-chunk-20-1.pdf}

\end{document}
